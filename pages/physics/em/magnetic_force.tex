\chapter{Magnetism}
Certain metals you find will exhibit what's known as magnetism. Magnetic objects, like electrical charges, will show levels of attractiveness towards "opposite charges". Magnetism, and it's forces and fields are directly related to electricity. First let's note the difference between magnetism and electricity. The first thing to note, all magnetic objects that have currently been observed are \textbf{dipoles} compare to singularly charged parties. They have a north and south ends. The north and south ends attract each other oppositely. If you break a magnet in half, it will just split into different north and south poles. Oddly enough our discussion of magnetism will start with the magnetic field.

\section{Magnetic Field}
A \textbf{magnetic field is created when a moving charged particle or current is present}. Remember that fields are a property of the surrounding space. There is going to be a magnetic field and electric field at the same time because if there is a charge present then of course you will have an electric field present. Now the difference is that a magnetic field is only present if the charge is moving or you have a current (moving charges). The electric field is always present and not dependent on if the charge is moving or not. The second thing to note is that the magnetic field exerts a force $\vec{\mathrm{F}}$ on other moving charges or currents in the space of the field. I don't know if at this point if we know that the magnetic field is self-interacting with the moving particle that creates it. 

\subsection{Properties of the Magnetic Force}
We note two things done by experiments to determine the properties of the magnetic force. First, that there is a direct relationship between strength of the magnetic force and the amount of charge present. Second, the force is solely perpendicular to the velocity of the charged particle and the magnetic field present. The cross-product shares the same properties, and we can write the Lorenz's Force for magnetism,
\begin{equation*}
	\vec{\mathrm{F_b}} = q\vec{v} \times \vec{\mathrm{{B}}} 
\end{equation*}

\pagebreak
\subsection{Magnetic Field Lines and Flux}
First, note that the magnetic field lines are not lines of force. The magnetic force is perpendicular to those field lines. Second, we note the Gauss's Law of Magnetic Flux, because there isn't a single monopole of magnetism discovered.  

\begin{equation*}
	\phi_B = \int \vec{\mathrm{B}} \cdot \diff{d\vec{S}} = 0 
\end{equation*}

\section{Motion in a Uniform Magnetic Field}
We consider a uniform magnetic field. Let's say that it points out in the direction of $-\hat{i}$ at all points in the region described by $x=a$ and $y=a$. There is a particle that enters the field with a certain velocity with a direction of positive $\hat{k}$. First with uniform magnetic field, the force is going to be centripetal. It's always going to be perpendicular to the field and the velocity. So we know we are going to have uniform rotational motion. By Lorenz's Force Law,
\begin{equation*}
	\vec{\mathrm{F_b}} = q\vec{v} \times \vec{\mathrm{B}}
\end{equation*}
We also know that for uniform rotational,
\begin{equation*}
	\mathrm{F}_b = \frac{mv^2}{R} = qv\mathrm{B}
\end{equation*}
We can get the radius of curvature from this equation, 
\begin{equation*}
	R = \frac{mv}{q\mathrm{B}}
\end{equation*}	
Now we can form our equations of motion, based on the fact that we have $R$. We know the equation of a circle, 
\begin{equation*}
	(x-h)^2 + (y-k)^2 = R^2
\end{equation*}
depending on how the circle is centered.

