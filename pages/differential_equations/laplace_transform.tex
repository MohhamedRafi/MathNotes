\chapter{Laplace Transform}
We note that a transform takes one function and converts it to another. The Laplace transform is an integral transform. Given any $f(x,y)$, by taking a derivative with respect to one variable will always lead to an equation (assuming the integral is definite and convergent) within terms of the other variable. 

Given a interval of integration bounded below, $[0, \infty)$, the integral transform is usually defined like this, 

\begin{equation*}
	\int_a^b K(s,t)f(t) \:\diff{dt} 
\end{equation*}

Where $f(t)$ is defined on the interval of integration. When the integral converges we will get a function that is defined in terms of $s$. We will look at a special case of the integral transform where $K = e^{-st}$. 

\section{Laplace Transform Integral}
Given $f(t)$, where it's defined on $[0,\infty)$, the Laplace Transform for this function is defined
\begin{equation*}
	\mathcal{L} \{ f(t) \} = \int_0^\infty e^{-st} f(t) \:\diff{dt} = F(s)
\end{equation*}
Note that computing the Laplace transform will always take some integration by parts, so to review a quick method of computing the integration by parts, check the integration methods section. Other than that, here is a table of general common Laplace Transforms.

\begin{align*}
	\mathcal{L} \{ 1 \} &= \frac{1}{s} \\
	\mathcal{L} \{ t^n \} &= \frac{n!}{s^{n+1}} \:\: n=1,2,3,...\\
	\mathcal{L} \{ e^{at} \} &= \frac{1}{s-a} \\
	\mathcal{L} \{ \sin(kt) \} &= \frac{k}{s^2+k^2} \\
	\mathcal{L} \{ \cos(kt) \} &= \frac{s}{s^2+k^2} \\
\end{align*}

\subsection{Conditions For Existence}
These conditions are not absolutely needed for the existence of a Laplace Transform but still can be used to see if one will exist. First $f(t)$ must be piecewise continuous. Basically continuous for piecewise functions. A general idea is that all the pieces are continuous, and the finite amount of discontinuities that exist are really just ones that are jumping. 

The formal definition is, on the interval $I$, $f$ is piecewise continuous if $0 \leq a \leq t \leq b$,
where $t_k$, $k=1,2,3,...,n$, and $t_{k-1} < t_k$ where at $f(t_k)$ exist a finite amount of discontinuities but are continuous on the open intervals of $(t_{k-1},t_k)$.

$f(t)$ also must be of exponential order, meaning that it doesn't grow faster then a exponential function. 
\begin{equation*}
	|f(t)| \leq Me^{ct}
\end{equation*}
where $ M > 0 $, $T>0$, and C is a positive constant, on the interval of $t>T$. 

\subsection{Laplace Inverse Transform}
Inverse transforms are simply reverses the transform, 

\begin{align*}
	\mathcal{L}^{-1} \{ \frac{1}{s} \} &= 1 \\
	\mathcal{L}^{-1} \{ \frac{n!}{s^{n+1}} \} &= t^n \:\: n=1,2,3,...\\
	\mathcal{L}^{-1} \{ \frac{1}{s-a} \} &= e^{at} \\
	\mathcal{L}^{-1} \{ \frac{k}{s^2+k^2} \} &= \sin(kt) \\
	\mathcal{L}^{-1} \{  \frac{s}{s^2+k^2} \} &= \cos(kt) \\
\end{align*}

The key is to algebraically simplify the expression or dividing or multiplying by a constant to make it look like one of the table functions. 

\begin{equation*}
	\mathcal{L}^{-1} \{\frac{1}{s^5}\}
\end{equation*}
We know that it looks like one of our table functions but it's missing the $n!$, so lets consider this
\begin{equation*}
	\frac{1}{n!}\mathcal{L}^{-1} {\frac{n!}{s^5}} = \mathcal{L}^{-1} \{\frac{1}{s^5}\}
\end{equation*}
Because $\mathcal{L}$ is a integral transform it has linearity with constants. We know the inverse of the right hand side, so we find the solution of 
\begin{equation*}
	\frac{1}{n!}\mathcal{L}^{-1} {\frac{n!}{s^5}} = \frac{1}{4!}t^4
\end{equation*}
	
\pagebreak
\section{Volterra Integral Equation}
Solve for $f(t)$ using this Volterra Integral equation by way of a Laplace Transform.

\begin{equation*}
f(t) = 4t-16\int_0^t \sin(t)f(t-\tau)\:\diff{d\tau}
\end{equation*}

First we apply the $\mathcal{L}$, 
\begin{align*}
	F &= \frac{4}{s^2} - 16\mathcal{L}\{\int_0^t \sin(t)f(t-\tau)\:\diff{d\tau}\} \\ 
	F &= \frac{4}{s^2} - 16\mathcal{L}\{\sin(t)\}F \\ 
	F + 16\frac{1}{s^2+1}F &= \frac{4}{s^2} \\ 
	F (1+\frac{16}{s^2+1}) &= F\frac{s^2+17}{s^2+1}=\frac{4}{s^2} \\ 
	F &= \frac{4}{s^2}\frac{s^2+1}{s^2+17} \\ 
	F &= 4(\frac{1}{s^2(s^2+17)}+\frac{s^2}{s^2(s^2+17)}) \\
	F &= 4(\frac{1}{s^2(s^2+17)}+\frac{1}{(s^2+17)}) \\
	F &= 4(\frac{\frac{1}{s(s^2+17)}}{s}+\frac{1}{s^2+17})
\end{align*}
Now we apply the $\mathcal{L}^{-1}$,
\begin{align*}
	f(t) &= 4\mathcal{L}^{-1}\{\frac{\frac{1}{s(s^2+17)}}{s}\} + 4\mathcal{L}^{-1}\{\frac{1}{s^2+17}\} \\
	f(t) &= 4\frac{1}{\sqrt{17}}\sin(\sqrt{17}t)+4\mathcal{L}^{-1}\{\frac{\frac{1}{s(s^2+17)}}{s}\}
\end{align*}
For the second inverse, we must first apply the second inverse convulsion theorem,
\begin{equation*}
\mathcal{L}^{-1}\{\frac{\frac{1}{s(s^2+17)}}{s}\} = \mathcal{L}^{-1}\{\frac{F_1(s)}{s}\} = \int_0^t f_1(\tau) \:\diff{d\tau} 
\end{equation*}
Where $f_1(t)=\mathcal{L}^{-1}\{F(s)\}$, by the first inverse convulsion theorem, 
\begin{equation*}
	f_1(t) =  \mathcal{L}^{-1} \{  \frac{1}{s}\frac{1}{s^2+17} \} = 
	 1 * (\frac{1}{\sqrt{17}}\sin(\sqrt{17}t)) = \int_0^t {\sqrt{17}}\sin(\sqrt{17}\tau) \:\diff{d\tau} 
\end{equation*}
\begin{equation*}
	f_1(t) = -\frac{1}{17}\cos(\sqrt{17}t)+\frac{1}{17}
\end{equation*}
\begin{align*}
	\mathcal{L}^{-1}\{\frac{\frac{1}{s(s^2+17)}}{s}\} &= \int_0^t -\frac{1}{17}\cos(\sqrt{17}\tau)+\frac{1}{17} \:\diff{d\tau} \\ 
	&= \frac{-1}{17\sqrt{17}}\sin(\sqrt{17}t)+\frac{1}{17}t \\ 
	f(t) &= 4\frac{1}{\sqrt{17}}\sin(\sqrt{17}t) + 4(\frac{-1}{17\sqrt{17}}\sin(\sqrt{17}t)+\frac{1}{17})= \frac{64\sqrt{17}}{289}\sin \:\left(\sqrt{17}t\right)+\frac{4t}{17}
\end{align*}

