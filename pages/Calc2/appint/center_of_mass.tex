\subsection{Center of Mass}
Center of mass can be very interesting and involved application of integrals. 

\paragraph*{To find the center of mass alongside} a line, we need to refer to Archimedes' Law of Lever: where a rod will be balanced, 
if $m_1d_1$ = $m_2d_2$, $m$ is the mass alongside the rod, and $d$ is the distance between the $\bar{x}$ and the mass, where that is the center posiition. 

\subparagraph*{We can write} $d$ in terms of the $\bar{x}$ and the position of the mass. 

\begin{align*}
	d_1 &= \bar{x} - x_1 \\ 
	d_2 &= x_2 - \bar{x} \\ 
	m_1(\bar{x} - x_1) &= m_2(x_2-\bar{x}) \\ 
	m_1\bar{x} - m_1x_1 &=m_2x_2-m_2\bar{x} \\ 
	m_1\bar{x}+m_2\bar{x} &= m_1x_1+m_2x_2 \\ 
	\bar{x} &= \frac{m_1x_1+m_2x_2}{m_1+m_2}
\end{align*}

\subparagraph*{We can now} try applying the above formula to the y-axis. Because, we're dealing with a plane, 
moments of can exist within alongside the y, or x-axis. These moments are where the object has the 
tendency of rotating around. We can calculate these momenets by taking the sum of the total of the point masses.

\begin{enumerate}
\item Moment of x-axis is equal to 
	\begin{equation}
		M_x = \sum_{i=0}^{n} m_iy_i 
	\end{equation}
\item Moment of y-axis is equal to 
	\begin{equation}
		M_y = \sum_{i=0}^{n} m_ix_i
	\end{equation}
\end{enumerate}

The center of mass is defined by a ordered pair: $(\bar{x}, \bar{y})$. Where the each is defined by it's respective moments, where $m_t = \sum m_i$

\begin{equation}
	C(\bar{x}, \bar{y}) \to \bar{x} = \frac{M_x}{m_t}, \bar{y} = \frac{M_y}{m_t}
\end{equation} 

As you can see, the center of mass is an average of the all the mass points inside this point defined plane. 

\paragraph {We can now} take this from dealing with point masses to an actual plane, defined by curves. 
Let's say we have two curves, $f(x), g(x)$, where $f(x) \geq g(x)$ on the interval of $[a, b]$. This plane, just to make it general, has different density at different points on it. The mass at some point is equal to the desntiy function times it's area, where $\delta(x)$ is the density function.

\begin{equation}
	M_i = \delta(x)(f(x)-g(x))
\end{equation}

Given that we are working with an interval, the total mass is defined by: 

\begin{equation}
	m_t = \int_a^b{\delta(x)(f(x)-g(x))} dx
\end{equation}

Now we have our total mass, we need to find our momements. Let's go back to what we where given in our moment equations in equation 1. 

\begin{equation*}
	M_x = my	
\end{equation*}

We know that, in a thin slice, that the average value is going to be the center point, so we can rewrite y in terms of that average. 

\begin{equation*}
	M_x = m (\frac{f(x)+g(x)}{2})
\end{equation*}

We can now plug that back into our integration formula,

\begin{align}
	M_x &= \int_a^b {\delta(x)(\frac{f(x)+g(x)}{2})(f(x)-g(x))}dx \\ 
	M_x &= \frac{1}{2}\int_a^b {\delta(x)(f(x)^2-g(x)^2)} dx
\end{align}

Now for the moment of y, 

\begin{align}
	M_y &= mx \\ 
	My  &= \int_a^b x\delta(x)(f(x)-g(x)) dx  
\end{align}

The three final equations we have for the center of mass problems are: 
\begin{align}
	m &= \int \delta(x) (f(x)-g(x)) dx\\
	\bar{x} &= \frac{M_y}{m} = \frac{1}{m}\int_a^b x\delta(x)(f(x)-g(x)) dx \\ 
	\bar{y} &= \frac{M_x}{m} = \frac{1}{2m} \int_a^b {\delta(x)(f(x)^2-g(x)^2)} dx 
\end{align}