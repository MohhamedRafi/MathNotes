\chapter{Electrostatics}
In electrostatics, the main force we deal with is the electrostatic force (Coulomb's force) that is gained from charged matters. Unlike gravity, when mass is presented, only attracts, there are two different types of charges. Positive and negatively charged matters. The charge is a quality of the proton and electron. Where net charge is the total charge in a body. Like most things, charge is conserved and obey the conservation principles.   

\section{Charge By Induction}
A object is only charged when is there is a net unbalance of charges. 
Meaning that if a object is uncharged it still has charge but the net sum of all the charges cancel out and evaluate to zero. Usually you cause this by moving some amount of electrons from one object to another. In physics, we consider protons to be the one that actually moves (all fields are defined by a positive moving charge), though in real life it's only the electron that moves. \textbf{Induction} is a method of charging a object without even coming into contact with one another. Say you have two uncharged spheres that are 
grounded by a insulating rod. That means they are completely isolated from the world. The spheres are in contact with each other. Then you bring a negatively charge rod close to, but not touching, one of the spheres. Then you move the sphere furthest away 
from the rod, away from the other sphere. The sphere closest to the rod gains a positive charge, the sphere furthest from the rod gains a negative charge. That's because, even though they weren't touching the sphere, the attraction between the negatively 
charged rod and the sphere's positively charged "protons" moved all the positive charges from both spheres into one sphere. Leaving the other sphere negatively charged. 

\section{Coulomb's Law}
There is a force between charges that there attracts or repeals. The force is defined by Coulomb's Law,

\begin{equation*}
	\vec{\mathbf{F}} = \frac{1}{4\pi\epsilon_0} \frac{|q_1q_2|}{r^2} \hat{r}
\end{equation*}
The abs matters as while $\vec{\mathbf{F}} \propto q_1q_2$, the direction is dependent on the signs of the charges. 

\subsection{Principle of Superposition}
The idea behind superposition is that we could evaluate the total force, field, or something at a certain point by just taking the superposition (or the sum) of all the total forces or fields at that point. For example, given $\mathbf{F_1,F_2,F_3}$ forces placed at some random points in space $\mathbb{R}^3$, the force on a charge $q$ is equal to $\sum \mathbf{F_qi}$. That seems natural and doesn't really need to be stated, but this principle allows us to ignore things when they don't suit our calculations at that time and evaluate them later in different parts. 

\subsection{Newton's Third Law}
In a system of charges, the simplest being two charges, we can see that as one charge feels some attraction or repulsive force from the other, that other charge will feel the same magnitude of force but in the opposite direction. This is true from a Newton's law's point of view and also the same from a Coulomb's law point of view. In Newton's corner, when one charge applies a force on another charge, that charge applies the same force on the other charge. Using Coulomb's law with the subscripts, 


	\begin{equation*}
		\mathbf{F}_{21} = \frac{1}{4\pi\epsilon_0} \frac{|q_1q_2|}{r^2} 
	\end{equation*}
	\begin{equation*}
		\mathbf{F}_{12} = \frac{1}{4\pi\epsilon_0} \frac{|q_1q_2|}{r^2} 
	\end{equation*}

Where it reads, the force on $2$ by $1$. The magnitude of the forces doesn't change depending on the perspective of who's doing what. 