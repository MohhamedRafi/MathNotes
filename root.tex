\documentclass[oneside]{mathnotes}
\usepackage{import}
\usepackage{xcolor}
\usepackage{ upgreek }
\usepackage{multicol}
\usepackage{mathtools}


\title{Math Notes}
\author{Mohhamed Rafi}

\graphicspath{{../MathNotes/pages/images/}}

\newcommand{\pdi}[2]{\frac{\partial #1}{\partial\mathrm{#2}}}
\newcommand\bm[1]{\begin{bmatrix}#1\end{bmatrix}} 
\newcommand{\diff}[1]{\mathrm{#1}}
\newcommand{\dd}[2]{\frac{\mathrm{d #1}}{\mathrm{d #2}}}
\newcommand{\mr}[1]{\mathrm{#1}}
\newcommand{\vv}[1]{\vec{\mathrm{#1}}}
\newcommand{\EMF}{\mathcal{E}}

\begin{document}

	\maketitle
	\tableofcontents\thispagestyle{fancy}
	\pagebreak

	\section*{Notational Notes}
	I don't use the regular notation for vector derivatives of their components. I find them confusing because sometimes I want to refer to the component by the subscript and not be mistaken for referring to the derivative.

	\begin{enumerate}
	\item $r'_x$ refers to the vector component of x after a derivative was taken.
	\item $r_x$ refers to the vector component of x.
	\end{enumerate}
	This shouldn't be confused with taking partial derivatives of these vectors. Note, you could still form a vector function with two variables. 

	\begin{equation*}
		\vec{\phi}(x,y) = \phi_x(x,y)\hat{i} + \phi_y(x,y)\hat{j} + \phi_z(x,y)\hat{k}
	\end{equation*}

	The partial $\partial_x\phi$ refers to taking the partial derivative with respect to $x$ to all vector components.


	% vector calculus
	\import{pages/calculus/}{line_int}
	\import{pages/calculus/}{surface_integrals}
	\import{pages/calculus/}{ftc_theorems}
	\import{pages/calculus/}{examples}
	
	% differential equations
	\import{pages/differential_equations/}{diff_eq}
	\import{pages/differential_equations/}{first_order_eq}
	\import{pages/differential_equations/}{homogeneousequations}
	\import{pages/differential_equations/}{different_methods}
	\import{pages/differential_equations/}{laplace_transform}
	\import{pages/differential_equations/}{apps}
	
	% physics 
	\import{pages/physics/em/}{coulomb_law}
	\import{pages/physics/em/}{electric_field}
	\import{pages/physics/em/}{electrostatics_kinematics}
	\import{pages/physics/em/}{electrical_energy}
	\import{pages/physics/em/}{electrical_current}
	% circuits 
	\import{pages/physics/em/}{magnetic_force}
	\import{pages/physics/em/}{magnetic_circuits}
	\import{pages/physics/em/}{em_induction}
	\import{pages/physics/em/}{inductance}
	% circuits 
	\import{pages/physics/em/}{circuits}



\end{document}