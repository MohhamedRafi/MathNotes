\subsection{Work}

Work is defined by a force applied over a distance.  

\begin{equation}
	W = Fd 
\end{equation}

Now we can define force in two different ways depending on the given units. If the problem is given weight in terms of pounds or Newtons, then it becomes a simple problem. 

\begin{equation}
	W = \int_a^b{w(x)(d)} dx
\end{equation}

Where $w(x)$ is the weight per slice, and $d$ is distance traveled. 

Now if the given units are in terms of density and volume. We need a new formula:

\begin{equation}
	W = g\delta\int_a^bV(d)dx
\end{equation}

\paragraph{Consider this problem}: a trough is shaped by the graph $y=x^6$. The cross sections are horizontal slices that run vertically up the trough. The trough is 4 meters tall and 5 meters wide. Water is filled in the trough up to 3m. The density of water is 1000 kg per m$^3$. If the water is being pumped from the top, much work is being done? Since these are slices are horizontal, we need the function in terms of y. 

\begin{align}
	y &= x^6 \\ 
	x &= \pm (y)^{1/6} \to x = 2y^{1/6}
\end{align}

This is the width of our trough as we run up the height, so $\Delta h = dy$. Our length of the trough is 5 meters so our volume is: 

\begin{equation}
	V = 5(2y^{1/6})\Delta h \to V = 10y^{1/6}\Delta h
\end{equation}

Force is equal to mass times accleration, which means that we need to apply $m=\delta V$. Then we apply our acceleration, which is $g$Now we can do our distance. 4 is the max height and it will travel by a height of y, so the distance travel is $4-y$. That means our work formula is 

\begin{equation}
	W = Fd = \delta V(4-y)g = g\delta (10y^{1/6})(4-y)\Delta h
\end{equation}
\begin{equation}
	W = 10\delta g\int_0^3(y^{1/6})(4-y)dy
\end{equation}