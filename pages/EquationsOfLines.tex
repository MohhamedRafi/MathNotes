\section{Equation of Lines}
Equations of lines are a simple introduction to how we are going to deal with functions that exist in three dimensional space. Typically, given 
a set of points, you can take those points and develop a position vector for each of those points. Using those position vectors, you can determine 
a direction vector. 

\subsection{Forms of Equation}
The first equation is called the vector function equation. 

\begin{align*}
	\vec{r}(t) = \vec{p_1} + (t)\vec{q}
\end{align*}

The second equation is called the parametric form: 

\begin{align*}
	x &= v_x t \\ 
	y &= v_y t \\ 
	z &= v_z t \\
\end{align*}

We get this form by solving the first vector equation for their individual components. 
By setting these equations together we get what is called the systematic equations:

\begin{align*}
	x^{-1}(v_x) = y^{-1}(v_y) = z^{-1}(v_z)
\end{align*}

\subsection{Example Problem: Create an equation of the line}
Given, $p_1(2,-4,1)$ and $p_2(0,4,-10)$, find the equation of line that passes through those points. 

\begin{align*}
	\vec{p_1} &= \begin{pmatrix} 
					2 \\ -4 \\ 1
				\end{pmatrix} \\ 
	\vec{p_2} &= \begin{pmatrix} 
					0 \\ 4 \\ -10
				\end{pmatrix}
\end{align*}

Using those position vectors, we can form a direction vector. 
\begin{align*}
	\vec{q} &= \vec{p_2} - \vec{p_1}\\
	\vec{q} &= \begin{pmatrix} 
				-2 \\ 8 \\ -11
			 \end{pmatrix}
\end{align*}

Now that we have a vector, we take the original position vector. Now scale it by a parameter $t$. 

\begin{align*}
	\vec{r} &= \vec{p_1} + t\vec{q} \\ 
	\vec{r} &= \begin{pmatrix}
					2-2t \\
					-4+8t \\
					1-11t  
				\end{pmatrix}
\end{align*}

\subsection{Example problem: Testing if a line is perpendicular or parallel}

Given, $p_1(2,0,9)$ and $p_2(-4, 1, -5)$, is the line through those points parallel, perpendicular, or neither to 
$\vec{r_i}(t) = \begin{pmatrix}
	5 \\ 1-9t \\ -8 - 4t 
\end{pmatrix}$. 

\linebreak

First we need an equation of the line through the first two points. 

\begin{align*}
	\vec{d} = \vec{p_2} - \vec{p_1} = \begin{pmatrix}
		-6 \\ 
		1 \\ 
		-14
	\end{pmatrix} 
\end{align*}

The equation of the line is

\begin{align*}
	\vec{r} &= \vec{p_1} + \vec{d}t  \\ 
	\vec{r} &= \begin{pmatrix}
		2-6t \\ 
		t \\ 
		9-14t	
	\end{pmatrix}
\end{align*} 

Now we can compare it to the original equation of the line. The position vector parallel to the original equation is just 
the coefficents of the $t$. But since those coefficents are not multiples of the coefficents of the new equation, the two lines are not parallel. 
To compare if they are perpendicular or not, we have to compute the dot product. The dot product will result in a non zero value, meaning that the two lines are not 
perpendicular to each other.

\subsection{Example problem: Testing if a line intersects another line} 

Determine if these two lines intersect each other. Given line $\vec{r}_1$ and line $\vec{r}_2$.

\begin{align*}
	\vec{r}_1 &= \begin{pmatrix} 
		8 + t \\ 
		5 + 6t \\ 
		4 - 2t 
	\end{pmatrix} 
\end{align*}

\begin{align*}
	\vec{r}_2 &= \begin{pmatrix}
		-7 + 12t \\ 
		3 - t \\ 
		14 + 8t 
	\end{pmatrix}
\end{align*}

We will need to use the parametric equations of lines to solve this problem. First we have to consider that $\vec{r}_1(t) \neq \vec{r}_2(t)$ for the same 
$t$. A note about parametric equations should be noted later. So we instead say $\vec{r}_1(t_1) = \vec{r}_2(t_2)$. Now we solve that system of  
equation.
\begin{align*}
	8 + t_1 &= -7 + 12t_2 \\
	5 + 6t_1 &= 3 - t_2 \\ 
	4-2t_1 &= 14 + 8t_2 	
\end{align*} 