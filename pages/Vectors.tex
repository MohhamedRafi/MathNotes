\section{Vectors}
Multivariable functions are functions that depend on many different independent variables. 
Multivariable calculus mainly deals with moving the concepts learned in single variable to multivariable. 
We are going to be dealing with multiple functions that have several independent variables. The first step is to figure out 
how we are going to deal with the new $N$-D space. The best way to deal with that is using vectors. 

\subsection{Simple Vector Notations}
We can use simple vectors to notate 3D points and graphs. This would be a point: 
\begin{align*}
	P_1= (x, y, z)
\end{align*}

Now a vector repersenting a position of this point would be called a position vector: 
\begin{align*}
	\vec{p} = \begin{pmatrix} x \\ y \\ z \end{pmatrix}
\end{align*}
Vectors have a defined magnitude and direction. The magnitude is computed and notated like this: 
\begin{align*}
	|| \vec{p} || = \sqrt{\sum^{n}_{i=0}{p_i^2}}
\end{align*}
Where $n$ is the number of components the vector has. Now is we take the magnitude and divide the vector
components by it's mangitude, we turned the vector into an unit vector. 
\begin{align*}
	\hat{p} = \frac{\vec{p}}{|| \vec{p} ||}
\end{align*}
This retains it's direction but has a mangitude of one. Now this is useful for making some computation easier. 
Vectors can be written in standard form, with their respect components, $\bf{\hat{i}}, \bf{\hat{j}}, \bf{\hat{k}}$.
\begin{align*}
	\vec{p} = a\hat{i} + b\hat{j} + c\hat{k}
\end{align*}

\subsection{Dot Product}
The dot product is called the scaler multiplication of two vectors in $\mathbb{R}^{n}$. Meaning that the dot product can be 
used in any space if the two vectors share the same amount of components. The dot product takes two vectors and returns a scalar product. 
That product can be use to determine if the two vectors are parallel or perpendicular.

\begin{center} \tikzset{every picture/.style={line width=0.75pt}} %set default line width to 0.75pt        

\begin{tikzpicture}[x=0.75pt,y=0.75pt,yscale=-1,xscale=1]
%uncomment if require: \path (0,495); %set diagram left start at 0, and has height of 495

%Shape: Axis 2D [id:dp41063265215967726] 
\draw  (209,307) -- (519,307)(290,98) -- (290,378) (512,302) -- (519,307) -- (512,312) (285,105) -- (290,98) -- (295,105)  ;
%Straight Lines [id:da024799959081051703] 
\draw    (290,307) -- (409.43,212.24) ;
\draw [shift={(411,211)}, rotate = 501.57] [color={rgb, 255:red, 0; green, 0; blue, 0 }  ][line width=0.75]    (10.93,-3.29) .. controls (6.95,-1.4) and (3.31,-0.3) .. (0,0) .. controls (3.31,0.3) and (6.95,1.4) .. (10.93,3.29)   ;

%Straight Lines [id:da17626829378696507] 
\draw    (290,307) -- (327.58,130.96) ;
\draw [shift={(328,129)}, rotate = 462.05] [color={rgb, 255:red, 0; green, 0; blue, 0 }  ][line width=0.75]    (10.93,-3.29) .. controls (6.95,-1.4) and (3.31,-0.3) .. (0,0) .. controls (3.31,0.3) and (6.95,1.4) .. (10.93,3.29)   ;


% Text Node
\draw (384,257) node   {$\vec{a}$};
% Text Node
\draw (332,166) node   {$\vec{b}$};
% Text Node
\draw (311,271) node   {$\theta $};


\end{tikzpicture}
 \end{center}

Now the dot product can be computed in multiple ways:

\begin{align*}	
	\vec{P} \cdot \vec{B} &= \sum^{n}_{i=0} (\vec{P}_i)(\vec{B}_i) \\ 
	\vec{P} \cdot \vec{B} &= (|| \vec{P} || \cdot || \vec{B} || )cos(\theta)
\end{align*}

These two formulas can be used together to determine if the dot product is parallel or perpendicular. 
Consider if the dot product is zero. Solving for $\theta$ would result in a value of $\frac{\pi}{2}$. This would mean that the two vectors
are perpendicular to each other. The parallel case is harder. 
A vector facing in the opposite direction can still be parallel. So that means that angle between the two vectors can be $\pi$ or $0$. Reversing the equation will result in: 

\begin{align*}
	\theta = cos^{-1}(\frac{\vec{P} \cdot \vec{B}}{|| \vec{P} || \cdot || \vec{B} ||})
\end{align*}

Now, another way to test if the vectors are parallel is using the unit vector, $\hat{v}$.